\documentclass{article}
\begin{document}

\title{
Development of Real-Time Systems\\
Assignment 3
}
\author{Elyasin Shaladi}
\maketitle

\tableofcontents

\section{Theory assignment}
The following part of assignment is a purely theoretical task that requires no additional tools.
The task is to find the largest possible frame size for the cyclic structured scheduler by
following requirements 1,2 and 3 for finding the largest frame size.

\subsection{Task set 1 --- T1(15, 1, 14) T2(20, 2, 26) T3(22, 3)}

\begin{tabular}{|l||l|l|l|}
\hline
Task   & $p_i$ & $e_i$ & $D_i$ \\
\hline
\hline
Task 1 & 15 & 1 & 14 \\
\hline
Task 2 & 20 & 2 & 26 \\
\hline
Task 3 & 22 & 3 & 22 \\
\hline
\end{tabular}

\subsubsection{Requirement 1}
$f$ must be greater or equal than the longest execution time, i.e. $f \geq 3$

\subsubsection{Requirement 2}
$f$ must evenly divide the hyperperiod $H$. The hyperperiod $H$ is the least common multiple of
all the periods in the task set, which are $15$, $20$ and $22$. This means $H = 660$. \\
Therefore possible values for $f$ are: $f \in \{1,2,3,4,5,6,10,11,15,22\}$

\subsubsection{Requirement 3}
The frame size $f$ has to fulfill the following requirement $\forall i: 2*f-gcd(p_i,f) \leq D_i$ \\
This requirement does not hold for $10 \leq f \leq 22$. \\
However this requirement holds for $1 \leq f \leq 6$ and thus $f = 6$ is the
largest possible frame size. \\
For $f = 6$ all three requirements are met.


\subsection{Task set 2 --- T1(4, 1) T2(5, 2, 7) T3(20, 5)}

\begin{tabular}{|l||l|l|l|}
\hline
Task   & $p_i$ & $e_i$ & $D_i$ \\
\hline
\hline
Task 1 & 4 & 1 & 4 \\
\hline
Task 2 & 5 & 2 & 7 \\
\hline
Task 3 & 20 & 5 & 20 \\
\hline
\end{tabular}

\subsubsection{Requirement 1}
$f$ must be greater or equal than the longest execution time, i.e. $f \geq 5$

\subsubsection{Requirement 2}
$f$ must evenly divide the hyperperiod $H$. The hyperperiod $H$ is the least common multiple of
all the periods in the task set, which are $4$, $5$ and $20$. This means $H = 20$. \\
Therefore possible values for $f$ are: $f \in \{1, 2, 4, 5, 10, 20\}$

\subsubsection{Requirement 3}
The frame size $f$ has to fulfill the following requirement $\forall i: 2*f-gcd(p_i,f) \leq D_i$ \\
This requirement does not hold for $5 \leq f \leq 20$. \\
However this requirement holds for $1 \leq f \leq 4$ and thus $f = 4$ is the
largest possible frame size. \\
\\
However, $f = 4$ does not fulfill the requirement $1$.
Therefore Task 3, which has an execution time of $5$,
has to be sliced into sub jobs as suggested in the lecture.


\subsection{Task set 3 --- T1(5, 0.1) T2(7, 1) T3(12, 6) T4(45, 9)}

\begin{tabular}{|l||l|l|l|}
\hline
Task   & $p_i$ & $e_i$ & $D_i$ \\
\hline
\hline
Task 1 & 5 & 0.1 & 5 \\
\hline
Task 2 & 7 & 1 & 7 \\
\hline
Task 3 & 12 & 6 & 12 \\
\hline
Task 4 & 45 & 9 & 45 \\
\hline
\end{tabular}

\subsubsection{Requirement 1}
$f$ must be greater or equal than the longest execution time, i.e. $f \geq 9$

\subsubsection{Requirement 2}
$f$ must evenly divide the hyperperiod $H$. The hyperperiod $H$ is the least common multiple of
all the periods in the task set, which are $5$, $7$, $12$ and $45$. This means $H = 1260$. \\
Therefore possible values for $f$ are: $f \in \{1, 2, 3, 4, 5, 6, 7, 9, 12, 15, 45\}$

\subsubsection{Requirement 3}
The frame size $f$ has to fulfill the following requirement $\forall i: 2*f-gcd(p_i,f) \leq D_i$ \\
This requirement does not hold for $4 \leq f \leq 45$. \\
However this requirement holds for $1 \leq f \leq 3$ and thus $f = 3$ is the
largest possible frame size. \\
\\
However, $f = 3$ does not fulfill the requirement $1$.
Therefore Task 3, which has an execution time of $6$, and Task 4, which has an execution time of $9$,
have to be sliced into sub jobs as suggested in the lecture.

\section{Simulation assignment}

The assignment is to use a real-time simulator to verify feasibility of a set of tasks.

\subsection{Simulation 1 --- RM Scheduler}

\begin{tabular}{|l||l|l|l|}
\hline
Task   & $p_i$ & $e_i$ & $D_i$ \\
\hline
\hline
Task 1 & 2 & 0.5 & 2 \\
\hline
Task 2 & 3 & 1.2 & 3 \\
\hline
Task 3 & 6 & 0.5 & 6 \\
\hline
\end{tabular}

\subsubsection{What is the utilization factor of the system and what is the value for Urm(3)?}
The utilization $U$ is $0.741$ according to SimSo (CPU total load). \\
Although according to the formula learned in the lecture the value is $0.733$,
but this does not change the feasibility test conclusion. \\
The value for $U_{rm}(3)$ is $0.780$. \\
\\
Since $U \leq U_{rm}(3)$ the system is guaranteed feasible.

\subsubsection{What is the minimum/maximum/average response time of all tasks?}

\begin{tabular}{|l||l|l|l|}
\hline
Task   & $min$ & $avg$ & $max$ \\
\hline
\hline
Task 1 & 0.5 & 0.5 & 0.5 \\
\hline
Task 2 & 1.7 & 1.7 & 1.7 \\
\hline
Task 3 & 2.7 & 2.7 & 2.7 \\
\hline
\end{tabular}

\subsubsection{Is any task missing the deadline? Which task? Where?}
No deadline is missed in this schedule.

\subsubsection{If a deadline is missed, could it be avoided by changing the scheduler?}
No deadline is missed in this schedule.


\subsection{Simulation 2 --- EDF Scheduler}

\begin{tabular}{|l||l|l|l|}
\hline
Task   & $p_i$ & $e_i$ & $D_i$ \\
\hline
\hline
Task 1 & 2 & 0.5 & 1.9 \\
\hline
Task 2 & 5 & 2 & 5 \\
\hline
Task 3 & 1 & 0.1 & 0.5 \\
\hline
Task 4 & 10 & 5 & 20 \\
\hline
\end{tabular}

\subsubsection{What is the utilization factor of the system and what is the value for Urm(3)?}
The utilization $U$ is $1$ according to SimSo (CPU total load). \\
Although according to the formula learned in the lecture the value is $1.25$.

The value for $U_{rm}(4)$ is $0.757$. Although I don’t see the utility of calculating $U_{rm}$, because
\begin{enumerate}
\item the scheduler is not an RM scheduler and
\item the utilization is greater than $1$, so the system is not feasible anyway.
\end{enumerate}

\subsubsection{What is the minimum/maximum/average response time of all tasks?}

\begin{tabular}{|l||l|l|l|}
\hline
Task   & $min$ & $avg$ & $max$ \\
\hline
\hline
Task 1 & 0.6 & 0.6 & 0.6 \\
\hline
Task 2 & 2.8 & 3.1 & 3.4 \\
\hline
Task 3 & 0.1 & 0.1 & 0.1 \\
\hline
Task 4 & 20 & 20 & 20 \\
\hline
\end{tabular}

\subsubsection{Is any task missing the deadline? Which task? Where?}
Task 4 meets the first deadline ($t = 20$),
but misses all subsequent deadlines ($t = 30, 40 , 50, 60, 70, 80, 90, 100$)

\subsubsection{If a deadline is missed, could it be avoided by changing the scheduler?}
No, I tried the available schedulers that we know from the class:
RM (Rate Monotonic, FP Fixed Priority and EDF Earliest Deadline First)
and none of them was feasible. \\
This is consistent with the finding that the utilization $U$ is greater than $1$ and thus the system is guaranteed not feasible. \\
\textsl{I also found (in a research paper) that the EDF scheduling algorithm is optimal.
I.e. if a real-time task set cannot be scheduled by EDF,
then this task set cannot be scheduled by any algorithm.}


\end{document}
